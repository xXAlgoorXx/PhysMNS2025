\section{Quantum Mechanics}
\begin{equation}
    p = m \cdot v
\end{equation}
\subsection{Schrödinger equation}
The time-independent Schrödinger equation:
\begin{equation}
    \label{eq:time-independent-schroedinger}
\left(-\frac{\hbar^2}{2m} \frac{d^2}{dx^2} + V(x)\right) \psi(x) = E \psi(x),
\end{equation}

\(\psi(x)\) is a probabilty amplitude which doesnt say much itself. Only \(|\psi(x)|^2\) which is a probabilty density.
\cref{eq:time-independent-schroedinger} can be transformed to 
\begin{equation}
    \frac{d^2}{dx^2} \psi(x) - b^2 \cdot \psi(x) = 0
\end{equation}
which can be solved by
\begin{equation}
    \psi(x) = A_1 \cdot e^{b \cdot x} + A_2 \cdot e^{-b \cdot x}
\end{equation}
\begin{equation}
    b = \sqrt{\frac{2m}{\hbar^2}\cdot (V-E)}
\end{equation}
If \(V > E\), then \(b\) is imaginary and the solution is
\begin{equation}
    \psi(x) = A_1 \cdot \sin(k \cdot x - \varphi)
\end{equation}
with
\begin{equation}
    k =\frac{2\pi}{\lambda} =  \sqrt{\frac{2m}{\hbar^2}\cdot (E-V)}.
\end{equation}
Else if \(V < E\), then \(b\) is real and the solution is
\begin{equation}
    \psi(x) = A_2 \cdot e^{-b \cdot x}
\end{equation}
which is a real exponential decay.

Heisenberg's uncertainty principle:
\begin{equation}
    \Delta x \cdot \Delta p \geq \frac{\hbar}{2}\quad \text{or}\quad \Delta E \cdot \Delta t \geq \hbar
\end{equation}

\subsection{Potenial Well}
